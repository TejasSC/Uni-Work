\documentclass[12pt]{report}
\usepackage{graphicx}
\title{COMP24412 Lab 3 - Grammars, Parsing and Logic}
\author{Tejas Chandrasekar}
\begin{document}
\maketitle
\newpage

\section{Building a CFG}
Steel is an alloy. Steel contains carbon.\par
\noindent POS tags:
\begin{itemize}
  \item NNP: Proper noun, singular
  \begin{itemize}
    \item steel
  \end{itemize}
  \item VBZ: Verb, 3rd person singular present
  \begin{itemize}
    \item is
    \item contains
  \end{itemize}
  \item NN: Noun, singular or mass
  \begin{itemize}
    \item alloy
    \item carbon
  \end{itemize}
  \item DT: Determiner
  \begin{itemize}
    \item an
  \end{itemize}
\end{itemize}
Phrasal nodes: Steel, an alloy (alloy), carbon.
CFG for the sentence:
\begin{itemize}
  \item s $\rightarrow$ np vp
  \item np $\rightarrow$ n
  \item vp $\rightarrow$ v det n
  \item vp $\rightarrow$ v n
  \item det $\rightarrow$ an
  \item v $\rightarrow$ is
  \item v $\rightarrow$ contains
  \item n $\rightarrow$ Steel
  \item n $\rightarrow$ alloy
  \item n $\rightarrow$ carbon
\end{itemize}

\section{Building a CFG in Prolog: using difference lists, and DFGs}
To enumerate all the words, I ran the prolog query sentence(X,[]).
See Prolog programs called ex2 and ex3.
\end{document}
