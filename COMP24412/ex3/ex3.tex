\documentclass{article}
\usepackage{graphicx}
\title{COMP24412 Lab 3 - The Riddle of Steel Report}
\author{Tejas Chandrasekar}
\begin{document}
\maketitle
\newpage

\section{Lexical Analysis}
Steel is alloy of iron and carbon, and sometimes other elements. Because of
its high tensile strength and low cost, it is a major component used in
buildings, infrastructure, tools, ships, automobiles, machines, appliances, and
weapons.\par
\begin{enumerate}
  \item POS tagging the above sentence:
  \begin{itemize}
    \item NN: Noun, singular or mass
    \begin{itemize}
      \item steel
      \item alloy
      \item iron
      \item carbon
      \item strength
      \item cost
      \item component
      \item infrastructure
    \end{itemize}
    \item NNS: Noun, plural
    \begin{itemize}
      \item elements
      \item buildings
      \item tools
      \item ships
      \item automobiles
      \item machines
      \item appliances
      \item weapons
    \end{itemize}
    \item JJ: Adjectives
    \begin{itemize}
      \item other
      \item high
      \item tensile
      \item low
      \item major
    \end{itemize}
    \item RB: Adverb
    \begin{itemize}
      \item sometimes
    \end{itemize}
    \item CC: Coordinating conjunction
    \begin{itemize}
      \item Because
      \item and
    \end{itemize}
    \item DT: Determiner
    \begin{itemize}
      \item an
      \item a
      \item its
    \end{itemize}
    \item IN: Preposition/subordinating conjunction
    \begin{itemize}
      \item in
      \item of
    \end{itemize}
    \item VBD: Verb, past tense
    \begin{itemize}
      \item Used
    \end{itemize}
    \item VZD: Verb, 3rd person singular present
    \begin{itemize}
      \item is
    \end{itemize}
    \item PP\$: Possessive pronoun
    \begin{itemize}
      \item it
    \end{itemize}
  \end{itemize}
  \item Pronominal co-references: steel, it
\end{enumerate}

\section{C-Structures}
Sentence is 'Steel is an alloy of iron and carbon, and sometimes other
elements.'.
\begin{enumerate}
  \item Constituency structure: see Figure 1.
  \begin{figure}
    \includegraphics[width=\linewidth]{task2img.PNG}
    \caption{C-strcuture for the sentence}
    \label{Cstruct}
  \end{figure}
  \item Nominal phrasal nodes:
  \item Co-ordinations:
  \begin{itemize}
    \item Coordinating conjunction: '...iron AND carbon, AND...'
    \item Gapped coordination: '...sometimes other elements.'
  \end{itemize}
\end{enumerate}

\section{Dependencies - Exploring new territories}
\begin{enumerate}
  \item Constituency parsing divides text into sub-phrases. Using a tree
  structure, the types of phrases belong on branches, the individual words in
  the sentence are leaves, and the designation 'Sentence' is the root.\par

  \noindent Dependency parsing connects words according to their relationships.
  Again using a tree structure, each node in the tree represents a word, with
  the leaf nodes being 'dependent' on the internal nodes, which are most often
  verbs.
  \item Dependency structure: see Figure 2.
  \begin{figure}
    \includegraphics[width=\linewidth]{task3dep.PNG}
    \caption{Dependency structure for sentence: 'Steel is an alloy of iron and
    carbon'}
    \label{DepStruct}
  \end{figure}
\end{enumerate}

\section{Open IE - Semantics}
\begin{enumerate}
  \item Here are the predicate-argument structures for the following sentences:
  \begin{enumerate}
    \item 'Steel is an alloy.': Entity(Steel) $\leftarrow$ Subject --
    relation(is) -- Object $\rightarrow$ (an alloy) Entity(alloy).
    \item 'Steel contains carbon.': Entity(Steel) $\leftarrow$ Subject --
    relation(contains) -- Object $\rightarrow$ (carbon) Entity(carbon).
    \item 'Steel contains iron.': Entity(Steel) $\leftarrow$ Subject --
    relation(contains) -- Object $\rightarrow$ (iron) Entity(iron).
  \end{enumerate}
  \item Prolog translation of triples:\par
    steel(X) :- alloy(X), contains(X, carbon), contains(X, iron).
  \item RDF translation of triples:
  \begin{enumerate}
    \item $\langle Steel \rangle$ $\langle is an \rangle$
    $\langle alloy \rangle$ .
    \item $\langle Steel \rangle$ $\langle contains \rangle$
    $\langle carbon \rangle$ .
    \item $\langle Steel \rangle$ $\langle contains \rangle$
    $\langle iron \rangle$ .
  \end{enumerate}
  \item Axiom formalisation using Description Logics: \par
    Steel$\equiv$ alloy $\sqcap$ hasCarbon $\sqcap$ hasIron
  \item 'Steel is an alloy of iron and carbon.' - as per the next figure, the
  predicate-argument structure is more complex, and the Prolog itself might
  differ like so:\par
    steel(X) :- alloy(X, [carbon, iron]).\par

    Previously, the three separate statements simply translated to three
    relational statements. However, an 'alloy of' some elements means there
    might be a predicate named alloy taking two arguments, testing if the first
    contains the second (which is in list form since an alloy can have more
    than two elements in it).\par
  \begin{figure}
    \includegraphics[width=\linewidth]{task4pas.PNG}
    \caption{Predicate-argument structure for: 'Steel is an alloy of iron and
    carbon.'}
    \label{PAStruct}
  \end{figure}
\end{enumerate}
\end{document}
